\documentclass[12pt, a4paper]{article}
\usepackage[utf8]{inputenc}

\title{Gee Up Calc}
\author{Simeon Borko}
\begin{document}
\maketitle

\section{Úvod}

\textit{Gee Up Calc} je kalkulačka s týmito operáciami:
\begin{itemize}
  \item +
  \item -
  \item *
  \item /
  \item prirodzená mocnina
  \item všeobecná odmocnina
  \item faktoriál
  \item prirodzený logaritmus
\end{itemize}

\section{Deb balíček}

Inštalácia: \texttt{sudo dpkg -i geeupcalc.deb} \\
Odinštalácia: \texttt{sudo dpkg -r geeupcalc.deb}

\section{Manuálna (od)inštalácia}
Ak nechcete využiť inštaláciu pomocou deb balíčku, využite tento postup:

\subsection{Závislosti}
Pred inštaláciou aplikácie je potrebné nainštalovať python3.5 a modul PyQt5 spustením príkazov:
\texttt{sudo apt install python3.5 python3-pip \\
sudo pip3 install PyQt5}

\subsection{Inštalácia}
Z priečinku aplikácie zadajte tieto príkazy: \\
\texttt{sudo mkdir /opt/GeeUpCalc \\
sudo cp -r * /opt/GeeUpCalc}

Aby ste vytvorili položku v menu, vytvorte súbor \\
\texttt{~/.local/share/applications/geeupcalc.desktop} \\
a vložte doňho tento obsah:

\begin{verbatim}
[Desktop Entry]
Encoding=UTF-8
Name=Gee Up Calc
Comment=Kalkulacka s odmocninou a prirodzenym logaritmom
Exec=python3.5 /opt/GeeUpCalc/src/Kalkulacka.py
Icon=/opt/GeeUpCalc/src/geeupcalc_icon.png
Categories=Utility;Calculator;
Version=1.0
Type=Application
Terminal=0
\end{verbatim}

A nakoniec vytvorte zástupcu na ploche: \\
\texttt{cp ~/.local/share/applications/geeupcalc.desktop `xdg-user-dir DESKTOP`}

\subsection{Odinštalácia}

Aplikáciu odstránite takto:
\begin{verbatim}
sudo rm -r /opt/GeeUpCalc/
rm ~/.local/share/applications/geeupcalc.desktop
rm `xdg-user-dir DESKTOP`/geeupcalc.desktop
\end{verbatim}

\end{document}
